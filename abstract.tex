Unique problems can be solved with similar solutions in software engineering. Within this paper, we will examine a research paper by Salvaneschi et al. \cite{7827078} which compares the recurring design pattern of the Observer \cite{gamma1995design} and the paradigm of reactive programming, both applied on the problem domain of reactive applications. Our review will focus on various aspects: After the evaluation of how design sciences \cite{wieringa} and empirical research principles were applied by the research group, we will challenge their research documentation with the FAIR research principles \cite{wilkinson:2016} for computer science research projects \cite{2019arXiv190805986H}. We will transfer the results by Salvaneschi et al. to the context of frontend engineering and answer a set of software engineering specific research questions thereafter. An outlook suggesting relevant topics for future research in the area of frontend engineering and reactive programming will conclude this paper.