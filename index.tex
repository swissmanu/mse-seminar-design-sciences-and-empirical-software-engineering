\documentclass[12pt,a4paper]{article}
\usepackage{ccicons}
\usepackage{hyperref}
\usepackage{minted}

\setminted{autogobble=true, tabsize=2, linenos=true, frame=single, breaklines=true}

\newcommand{\version}{v0.1}


\begin{document}
\begin{centering}
	\Huge{tbd. Design Sciences and Empirical Software Engineering / Reactive Programming}
	\par
	\vspace{2ex}

	\normalsize{
		Manuel Alabor\\
		Supervised by Prof. Dr. Olaf Zimmermann\\
		\par
		\vspace{2ex}
		HSR University of Applied Sciences Rapperswil\\
		\par
		\vspace{2ex}
		\today{} (\version)
	}
	\par
	\vspace{2ex}

	\begin{quotation}
		\small{
			\textbf{Abstract}---A summary of this paper within 300 words to give the interested reader an overview of what they can expect.
		}
		\par
		\vspace{2ex}

		\small{
			\textbf{Keywords}: tbd.
		}
		\par
		\vspace{4ex}
	\end{quotation}
\end{centering}

\section{Introduction}
Solving unique problems through the application of software engineering principles does not automatically lead to unique solutions in general. In fact, the emergence of recurring design patterns can be observed. Salvaneschi et al. \cite{7827078} challenge the observer pattern \cite{gamma1995design} and its impact on source code comprehensibility in comparison to the usage of reactive programming \emph{(RP)}.

We will examine the paper \emph{On the Positive Effect of Reactive Programming on Software Comprehension: An Empirical Study} by Salvaneschi et al. \cite{7827078} and answer two sets of questions within this paper. With the focus on  design sciences and empirical software engineering methods, set \textbf{Q1} considers the following three questions:

\begin{itemize}
	\item \textbf{Q1.1}: Which empirical research methods, approaches and concepts were applied?
	\item \textbf{Q1.2}: Are these research methods, approaches and concepts applied well, what could have been done better?
	\item \textbf{Q1.3}: Does the paper meet FAIR principles \cite{2019arXiv190805986H} \cite{wilkinson:2016}?
\end{itemize}

As the paper further compares RP with the observer pattern \cite{gamma1995design}, an equivalent based on object oriented programming \emph{(OOP)} paradigms, question set \textbf{Q2} reviews software engineering specific aspects of that comparison:

\begin{itemize}
	\item \textbf{Q2.1}: How does the paper define RP and OOP?
	\item \textbf{Q2.2}: When should RP be considered, and when not?
	\item \textbf{Q2.3}: Which experiments should be conducted in future work?
	\item \textbf{Q2.4}: Which design alternatives exist (in a particular application context)?
\end{itemize}

In the following we will introduce and clarify important terms. Section 2 and 3 will examine the mentioned paper more closely and answer \textbf{Q1} and \textbf{Q2.1}. The questions \textbf{Q2.2} through \textbf{Q2.4} will be tackled in section 4 where we discuss earlier findings.

\subsection{Empirical Software Engineering}
Software engineering applies engineering principles when working with software artifacts \cite{159342} throughout their life cycle. Although ``\emph{the application of a systematic, disciplined, quantifiable approach to the development (\dots) of software (\dots)}'' \cite{159342} may rightful restrain us methodically, it is agreed upon that the process leading to a software product is of creative nature \cite{8051350}. It is this creative environment that keep the fundamental building blocks of software development, maintenance and operation in a constant flux of change. As tools and methods keep changing \cite{kapferer:2019:empirical}, software engineering organizations need to adapt to new circumstances often. Kapferer \cite{kapferer:2019:empirical} highlights in his introduction to empirical software engineering how the exertion of continuous quality improvement procedures like \emph{Plan/Do/Study/Act} \cite{deming} or an agile project setup like \emph{Scrum} leads to the controlled adoption of such. A reliable benchmark allows an organization to estimate feasibility of a new method or tool in advance as well as measure impact after its implementation.

Hence software engineering problems lack analytical and scientific formalism, empirical software engineering applies empirical research methods in software engineering to yield a representative benchmark for a problem domain.

\subsection{Design Science}

Roel J. Wieranga derives guidelines specifically for software engineering research projects from design science in his book \emph{Design Science Methodology for Information Systems and Software Engineering} \cite{wieringa}. As Balestra \cite{balestra:2019:designscience:articactandcontext} summarizes in his paper about design science and their application in software projects, Wieranga \cite{wieringa} defines an iterative process consisting of a \emph{design} and \emph{investigation} step. During \emph{design}, an \emph{artifact} is designed to satisfy the goals of one or more \emph{stakeholders} under consideration of a specific \emph{knowledge context}. The artifact and context is then put under \emph{investigation} to answer knowledge questions. These answers lead to a feedback loop where gained insights get fed into another design iteration. This process might be repeated until the stakeholder goals are met at a satisfying level.

\subsection{FAIR Research Principles}

\subsection{Reactive Programming}
What is it? Why?

\subsection{Observer Pattern}
When dividing a large component into smaller sub components, we usually want these components to be as isolated as necessary so they become more reusable, hence target for loose coupling between them. Still, the new components need a way to interact with each other.

The observer pattern \cite{gamma1995design} defines an interface where \emph{observers} can subscribe to state change notifications of a \emph{subject}. Once notified, an \emph{observer} might query the \emph{subject} for its latest state and act accordingly to its implementation.

The \emph{EventTarget interface}\footnote{\url{https://dom.spec.whatwg.org/\#interface-eventtarget}} of the document object model \emph{(DOM)} API exposed to a browsers JavaScript runtime (among others) is an evolution of the observer pattern.

\begin{listing}[H]
	\mint{JavaScript}{window.addEventListener('click', e => console.log('click', e.target));}
	\caption{Add a click handler to the \mintinline{JavaScript}{window} event target (JavaScript)}
	\label{lst:eventtarget}
\end{listing}

Listing \ref{lst:eventtarget} shows how an event handler (the \emph{observer}) can register for events emitted by an event target (the \emph{subject}). All registered event handlers will be called one after another once the event target dispatches (\emph{notifies}) an event. The handler might gain access to the event targets state via the provided event object.


\section{Artifact Review}
Will not do: Review on statistic

I will assess the paper by Salvaneschi et al. \cite{7827078} in regards of the following aspects:
\begin{itemize}
	\item Applied scientific methods (foundation by Kapferer \cite{kapferer:2019:empirical} and others tbd.)
	\item Study setup
	\item Challenge with FAIR Principles \cite{2019arXiv190805986H} \cite{wilkinson:2016}
\end{itemize}



\section{Review Results}
Is the conclusion of Salvaneschi et al. \cite{7827078} reasonable and well argued?



\section{Discussion}
Given the results, I will put RP in context with current trends in software engineering, in particular in its subdiscipline frontend engineering.
Ideas: RxJS, Elm, ReactJS \dots tbd.

\section{Conclusion}
Summarize findings, give proposal for future work on topic.

\bibliographystyle{splncs04}
\bibliography{index}

\section*{License}
\ccby\thinspace\thinspace This work is licensed under a \href{https://creativecommons.org/licenses/by/4.0/}{Creative Commons Attribution 4.0 International License}.

\end{document}