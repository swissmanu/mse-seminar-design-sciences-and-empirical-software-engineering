\documentclass[12pt,a4paper,twocolumn]{article}
\usepackage{ccicons}
\usepackage{hyperref}

\begin{document}
\twocolumn[{
		\begin{centering}
			\huge{tbd. Design Sciences \& Empirical Software Engineering / Reactive Programming}
			\par
			\vspace{2ex}

			\large{
				Manuel Alabor\\
				Supervised by Prof. Dr. Olaf Zimmermann\\
				University of Applied Sciences Rapperswil
			}
			\par
			\vspace{2ex}

			\large{\today}
			\par
			\vspace{2ex}

			\small{
				\textbf{Abstract}---A summary of this paper within 300 words to give the interested reader an overview of what they can expect.
			}
			\par
			\vspace{2ex}

			\small{
				\textbf{Keywords}: tbd.
			}
			\par
			\vspace{4ex}
		\end{centering}
	}]

	\section{Introduction}
	Software engineering provides patterns to solve recurring problems with proved and tested solutions. However, it is rare that a given problem must always be solved with exactly \emph{one} solution. Salvaneschi et al. \cite{7827078} examine reactive programming \emph{(RP)} as an alternative to the observer pattern \cite{gamma1995design} and its impact on source code comprehensibility.

	\dots{more intro}

	\subsection{Reactive programming}
	What is it? Why?

	\subsection{Observer pattern}
	What is it? Why?

	\subsection{Design sciences \& empirical software engineering}
	What are those? Why in this paper?

	\section{Review}
	I will assess the paper by Salvaneschi et al. \cite{7827078} in regards of the following aspects:
	\begin{itemize}
		\item Applied scientific methods (foundation by Kapferer \cite{kapferer:2019:empirical} and others tbd.)
		\item Study setup
		\item Challenge with FAIR Principles \cite{2019arXiv190805986H} \cite{wilkinson:2016}
	\end{itemize}

	\section{Results}
	Is the conclusion of Salvaneschi et al. \cite{7827078} reasonable and well argued?

	\section{Discussion}
	Given the results, I will put RP in context with current trends in software engineering, in particular in its subdiscipline frontend engineering.
	Ideas: RxJS, Elm, ReactJS \dots tbd.

	\section{Conclusion}
	Summarize findings, give proposal for future work on topic.

	\bibliographystyle{splncs04}
	\bibliography{index}

	\section*{License}
	\ccby\thinspace\thinspace This work is licensed under a \href{https://creativecommons.org/licenses/by/4.0/}{Creative Commons Attribution 4.0 International License}.

\end{document}