\documentclass[11pt,a4paper,twocolumn]{article}
\usepackage{ccicons}
\usepackage{hyperref}

\begin{document}
\twocolumn[{
		\begin{centering}
			\huge{tbd. Design Sciences and Empirical Software Engineering / Reactive Programming}
			\par
			\vspace{2ex}

			\large{
				Manuel Alabor\\
				Supervised by Prof. Dr. Olaf Zimmermann\\
				University of Applied Sciences Rapperswil
			}
			\par
			\vspace{2ex}

			\large{\today}
			\par
			\vspace{2ex}

			\small{
				\textbf{Abstract}---A summary of this paper within 300 words to give the interested reader an overview of what they can expect.
			}
			\par
			\vspace{2ex}

			\small{
				\textbf{Keywords}: tbd.
			}
			\par
			\vspace{4ex}
		\end{centering}
	}]

	\section{Introduction}
	Software engineering uses engineering principles when working with a software artifact \cite{159342} throughout its life cycle. Although \emph{the application of a systematic, disciplined, quantifiable approach to the development (\dots) of software} \cite{159342} may restrain us methodically, it is agreed upon that the process leading to a solution is of creative nature \cite{8051350}.

	Solving unique problems through the application of software engineering principles does not automatically lead to unique solutions in general. In fact, the emergence of recurring design patterns can be observed. Salvaneschi et al. \cite{7827078} challenge the observer pattern \cite{gamma1995design} and its impact on source code comprehensibility in comparison to the usage of reactive programming \emph{(RP)}.

	We will examine the artifact \cite{balestra:2019:designscience:articactandcontext} \textbf{A1} \emph{On the Positive Effect of Reactive Programming on Software Comprehension: An Empirical Study} by Salvaneschi et al. \cite{7827078} and answer two sets of questions within this paper. \textbf{Q1} considers design sciences and empirical software engineering methods, approaches and concepts as knowledge context \cite{balestra:2019:designscience:articactandcontext}:

	\begin{itemize}
		\item \textbf{Q1.1}: Which empirical research methods, approaches and concepts were applied?
		\item \textbf{Q1.2}: Are these research methods, approaches and concepts applied well, what could have been done better?
		\item \textbf{Q1.3}: Does the paper meet FAIR principles \cite{2019arXiv190805986H} \cite{wilkinson:2016}?
	\end{itemize}

	As \textbf{A1} compares RP with the observer pattern \cite{gamma1995design}, an equivalent to RP in object oriented programming \emph{(OOP)}, \textbf{Q2} states questions under the software engineering context:

	\begin{itemize}
		\item \textbf{Q2.1}: How does the paper define RP and OOP?
		\item \textbf{Q2.2}: When should RP be considered, and when not?
		\item \textbf{Q2.3}: Which experiments should be conducted in future work?
		\item \textbf{Q2.4}: Which design alternatives exist (in a particular application context)?
	\end{itemize}

	In the following we will introduce and define important terms. Section 2 and 3 will examine \textbf{A1} and answer \textbf{Q1} and \textbf{Q2.1}. \textbf{Q2.2} through \textbf{Q2.4} will be tackled in section 4 where we discuss previous findings.

	\subsection{Design sciences and empirical software engineering}
	What are those? Why in this paper?

	\subsection{Reactive programming}
	What is it? Why?

	\subsection{Observer pattern}
	What is it? Why?

	\section{Artifact Review}
	Will not do: Review on statistic

	I will assess the paper by Salvaneschi et al. \cite{7827078} in regards of the following aspects:
	\begin{itemize}
		\item Applied scientific methods (foundation by Kapferer \cite{kapferer:2019:empirical} and others tbd.)
		\item Study setup
		\item Challenge with FAIR Principles \cite{2019arXiv190805986H} \cite{wilkinson:2016}
	\end{itemize}



	\section{Review Results}
	Is the conclusion of Salvaneschi et al. \cite{7827078} reasonable and well argued?



	\section{Discussion}
	Given the results, I will put RP in context with current trends in software engineering, in particular in its subdiscipline frontend engineering.
	Ideas: RxJS, Elm, ReactJS \dots tbd.

	\section{Conclusion}
	Summarize findings, give proposal for future work on topic.

	\bibliographystyle{splncs04}
	\bibliography{index}

	\section*{License}
	\ccby\thinspace\thinspace This work is licensed under a \href{https://creativecommons.org/licenses/by/4.0/}{Creative Commons Attribution 4.0 International License}.

\end{document}